\documentclass[12pt]{article}
\usepackage[a4paper,margin=1in,footskip=0.25in]{geometry} % set margins
\usepackage[portuguese]{babel}
\usepackage[utf8]{inputenc}
\usepackage{hyperref} 
\usepackage{amsmath}
\usepackage{amssymb}
\usepackage{amsthm}
\usepackage{graphicx}    % needed for include graphics
\usepackage{indentfirst}
\usepackage{float}       % needed for [H] figure placement option
\usepackage{setspace}    % needed for doublespacing
\usepackage{tikz}

% Macros
\renewcommand{\familydefault}{\sfdefault} % sans-serif
\newcommand{\lowtext}[1]{$_{\text{#1}}$}

% Adds ./img/ to the path of figures
\graphicspath{./img/}

\title{Relatório EP1 - MAC0219}
\author{Bruno Sesso, Gustavo Estrela de Matos, Lucas Sung Jun Hong}

\begin{document}
% Espaçamento duplo 
\doublespacing
\begin{titlepage}
    \vfill
    \begin{center}
        \vspace{0.5\textheight}
        \noindent
        Instituto de Matemática e Estatística \\
        EP1 - MAC0219 \\
        \vfill
        \noindent
        {\Large Cálculo do Conjunto de Mandelbrot
        em Paralelo com Pthreads e OpenMP} \\
        \bigskip
        \bigskip
        \begin{tabular}{ll}
            {\bf Professor:} & {Alfredo Goldman} \\
            {\bf Alunos:}    & {Bruno Sesso} \\
                             & {Gustavo Estrela de Matos} \\
                             & {Lucas Sung Jun Hong} \\
        \end{tabular} \\
        \vspace{\fill}
       \bigskip
        São Paulo, \today \\
       \bigskip
    \end{center}
\end{titlepage}

\pagebreak
\tableofcontents
\pagebreak

\section{Introdução}
\section{Código Sequencial}
\section{Código em OpenMP}
\section{Código em Pthreads}
\section{Discussões Gerais}
\section{Conclusão}
\end{document}


